% Options for packages loaded elsewhere
\PassOptionsToPackage{unicode}{hyperref}
\PassOptionsToPackage{hyphens}{url}
%
\documentclass[
]{article}
\usepackage{amsmath,amssymb}
\usepackage{lmodern}
\usepackage{ifxetex,ifluatex}
\ifnum 0\ifxetex 1\fi\ifluatex 1\fi=0 % if pdftex
  \usepackage[T1]{fontenc}
  \usepackage[utf8]{inputenc}
  \usepackage{textcomp} % provide euro and other symbols
\else % if luatex or xetex
  \usepackage{unicode-math}
  \defaultfontfeatures{Scale=MatchLowercase}
  \defaultfontfeatures[\rmfamily]{Ligatures=TeX,Scale=1}
\fi
% Use upquote if available, for straight quotes in verbatim environments
\IfFileExists{upquote.sty}{\usepackage{upquote}}{}
\IfFileExists{microtype.sty}{% use microtype if available
  \usepackage[]{microtype}
  \UseMicrotypeSet[protrusion]{basicmath} % disable protrusion for tt fonts
}{}
\makeatletter
\@ifundefined{KOMAClassName}{% if non-KOMA class
  \IfFileExists{parskip.sty}{%
    \usepackage{parskip}
  }{% else
    \setlength{\parindent}{0pt}
    \setlength{\parskip}{6pt plus 2pt minus 1pt}}
}{% if KOMA class
  \KOMAoptions{parskip=half}}
\makeatother
\usepackage{xcolor}
\IfFileExists{xurl.sty}{\usepackage{xurl}}{} % add URL line breaks if available
\IfFileExists{bookmark.sty}{\usepackage{bookmark}}{\usepackage{hyperref}}
\hypersetup{
  pdftitle={Data 605: Assignment 2},
  pdfauthor={Leticia Salazar},
  hidelinks,
  pdfcreator={LaTeX via pandoc}}
\urlstyle{same} % disable monospaced font for URLs
\usepackage[margin=1in]{geometry}
\usepackage{color}
\usepackage{fancyvrb}
\newcommand{\VerbBar}{|}
\newcommand{\VERB}{\Verb[commandchars=\\\{\}]}
\DefineVerbatimEnvironment{Highlighting}{Verbatim}{commandchars=\\\{\}}
% Add ',fontsize=\small' for more characters per line
\usepackage{framed}
\definecolor{shadecolor}{RGB}{248,248,248}
\newenvironment{Shaded}{\begin{snugshade}}{\end{snugshade}}
\newcommand{\AlertTok}[1]{\textcolor[rgb]{0.94,0.16,0.16}{#1}}
\newcommand{\AnnotationTok}[1]{\textcolor[rgb]{0.56,0.35,0.01}{\textbf{\textit{#1}}}}
\newcommand{\AttributeTok}[1]{\textcolor[rgb]{0.77,0.63,0.00}{#1}}
\newcommand{\BaseNTok}[1]{\textcolor[rgb]{0.00,0.00,0.81}{#1}}
\newcommand{\BuiltInTok}[1]{#1}
\newcommand{\CharTok}[1]{\textcolor[rgb]{0.31,0.60,0.02}{#1}}
\newcommand{\CommentTok}[1]{\textcolor[rgb]{0.56,0.35,0.01}{\textit{#1}}}
\newcommand{\CommentVarTok}[1]{\textcolor[rgb]{0.56,0.35,0.01}{\textbf{\textit{#1}}}}
\newcommand{\ConstantTok}[1]{\textcolor[rgb]{0.00,0.00,0.00}{#1}}
\newcommand{\ControlFlowTok}[1]{\textcolor[rgb]{0.13,0.29,0.53}{\textbf{#1}}}
\newcommand{\DataTypeTok}[1]{\textcolor[rgb]{0.13,0.29,0.53}{#1}}
\newcommand{\DecValTok}[1]{\textcolor[rgb]{0.00,0.00,0.81}{#1}}
\newcommand{\DocumentationTok}[1]{\textcolor[rgb]{0.56,0.35,0.01}{\textbf{\textit{#1}}}}
\newcommand{\ErrorTok}[1]{\textcolor[rgb]{0.64,0.00,0.00}{\textbf{#1}}}
\newcommand{\ExtensionTok}[1]{#1}
\newcommand{\FloatTok}[1]{\textcolor[rgb]{0.00,0.00,0.81}{#1}}
\newcommand{\FunctionTok}[1]{\textcolor[rgb]{0.00,0.00,0.00}{#1}}
\newcommand{\ImportTok}[1]{#1}
\newcommand{\InformationTok}[1]{\textcolor[rgb]{0.56,0.35,0.01}{\textbf{\textit{#1}}}}
\newcommand{\KeywordTok}[1]{\textcolor[rgb]{0.13,0.29,0.53}{\textbf{#1}}}
\newcommand{\NormalTok}[1]{#1}
\newcommand{\OperatorTok}[1]{\textcolor[rgb]{0.81,0.36,0.00}{\textbf{#1}}}
\newcommand{\OtherTok}[1]{\textcolor[rgb]{0.56,0.35,0.01}{#1}}
\newcommand{\PreprocessorTok}[1]{\textcolor[rgb]{0.56,0.35,0.01}{\textit{#1}}}
\newcommand{\RegionMarkerTok}[1]{#1}
\newcommand{\SpecialCharTok}[1]{\textcolor[rgb]{0.00,0.00,0.00}{#1}}
\newcommand{\SpecialStringTok}[1]{\textcolor[rgb]{0.31,0.60,0.02}{#1}}
\newcommand{\StringTok}[1]{\textcolor[rgb]{0.31,0.60,0.02}{#1}}
\newcommand{\VariableTok}[1]{\textcolor[rgb]{0.00,0.00,0.00}{#1}}
\newcommand{\VerbatimStringTok}[1]{\textcolor[rgb]{0.31,0.60,0.02}{#1}}
\newcommand{\WarningTok}[1]{\textcolor[rgb]{0.56,0.35,0.01}{\textbf{\textit{#1}}}}
\usepackage{graphicx}
\makeatletter
\def\maxwidth{\ifdim\Gin@nat@width>\linewidth\linewidth\else\Gin@nat@width\fi}
\def\maxheight{\ifdim\Gin@nat@height>\textheight\textheight\else\Gin@nat@height\fi}
\makeatother
% Scale images if necessary, so that they will not overflow the page
% margins by default, and it is still possible to overwrite the defaults
% using explicit options in \includegraphics[width, height, ...]{}
\setkeys{Gin}{width=\maxwidth,height=\maxheight,keepaspectratio}
% Set default figure placement to htbp
\makeatletter
\def\fps@figure{htbp}
\makeatother
\setlength{\emergencystretch}{3em} % prevent overfull lines
\providecommand{\tightlist}{%
  \setlength{\itemsep}{0pt}\setlength{\parskip}{0pt}}
\setcounter{secnumdepth}{-\maxdimen} % remove section numbering
\ifluatex
  \usepackage{selnolig}  % disable illegal ligatures
\fi

\title{Data 605: Assignment 2}
\author{Leticia Salazar}
\date{2/4/2022}

\begin{document}
\maketitle

{
\setcounter{tocdepth}{2}
\tableofcontents
}
\(~\)

\hypertarget{problem-set-1}{%
\subsubsection{Problem Set 1}\label{problem-set-1}}

\(~\)

\hypertarget{show-that-ata-ne-aat-in-general.-proof-and-demonstration.}{%
\paragraph{\texorpdfstring{(1) Show that \emph{\(A^TA \ne AA^T\)} in
general. Proof and
Demonstration.}{(1) Show that A\^{}TA \textbackslash ne AA\^{}T in general. Proof and Demonstration.}}\label{show-that-ata-ne-aat-in-general.-proof-and-demonstration.}}

\begin{Shaded}
\begin{Highlighting}[]
\CommentTok{\# Creating matrix 4 x 3 with random numbers}
\NormalTok{matrixA }\OtherTok{\textless{}{-}} \FunctionTok{matrix}\NormalTok{(}\FunctionTok{c}\NormalTok{(}\DecValTok{1}\SpecialCharTok{:}\DecValTok{12}\NormalTok{), }\AttributeTok{nrow =} \DecValTok{4}\NormalTok{, }\AttributeTok{byrow =}\NormalTok{ T)}
\NormalTok{matrixA}
\end{Highlighting}
\end{Shaded}

\begin{verbatim}
##      [,1] [,2] [,3]
## [1,]    1    2    3
## [2,]    4    5    6
## [3,]    7    8    9
## [4,]   10   11   12
\end{verbatim}

\(~\)

\begin{Shaded}
\begin{Highlighting}[]
\CommentTok{\# Transpose "matrixA"; it creates a 3 x 4 matrix}
\NormalTok{matrixAT }\OtherTok{\textless{}{-}} \FunctionTok{t}\NormalTok{(matrixA)}
\NormalTok{matrixAT}
\end{Highlighting}
\end{Shaded}

\begin{verbatim}
##      [,1] [,2] [,3] [,4]
## [1,]    1    4    7   10
## [2,]    2    5    8   11
## [3,]    3    6    9   12
\end{verbatim}

\(~\)

\begin{Shaded}
\begin{Highlighting}[]
\CommentTok{\# Checking to see if A\^{}TA ≠ AA\^{}T}

\CommentTok{\# A\^{}TA; matrixA multiplied by matrixAT}
\NormalTok{ATA }\OtherTok{\textless{}{-}}\NormalTok{ matrixA }\SpecialCharTok{\%*\%}\NormalTok{ matrixAT}
\NormalTok{ATA}
\end{Highlighting}
\end{Shaded}

\begin{verbatim}
##      [,1] [,2] [,3] [,4]
## [1,]   14   32   50   68
## [2,]   32   77  122  167
## [3,]   50  122  194  266
## [4,]   68  167  266  365
\end{verbatim}

\begin{Shaded}
\begin{Highlighting}[]
\CommentTok{\#AA\^{}T; matrixAT multiplied by matrixA}
\NormalTok{AAT }\OtherTok{\textless{}{-}}\NormalTok{ matrixAT }\SpecialCharTok{\%*\%}\NormalTok{ matrixA}
\NormalTok{AAT}
\end{Highlighting}
\end{Shaded}

\begin{verbatim}
##      [,1] [,2] [,3]
## [1,]  166  188  210
## [2,]  188  214  240
## [3,]  210  240  270
\end{verbatim}

\(~\)

\hypertarget{for-a-special-type-of-square-matrix-a-we-get-ata-aat.-under-what-conditions-could-this-be-true-hint-the-identity-matrix-i-is-an-example-of-such-a-matrix.}{%
\paragraph{\texorpdfstring{(2) For a special type of square matrix A, we
get \emph{\(A^TA = AA^T\)}. Under what conditions could this be true?
(Hint: The Identity matrix \emph{I} is an example of such a
matrix).}{(2) For a special type of square matrix A, we get A\^{}TA = AA\^{}T. Under what conditions could this be true? (Hint: The Identity matrix I is an example of such a matrix).}}\label{for-a-special-type-of-square-matrix-a-we-get-ata-aat.-under-what-conditions-could-this-be-true-hint-the-identity-matrix-i-is-an-example-of-such-a-matrix.}}

\begin{Shaded}
\begin{Highlighting}[]
\CommentTok{\# Create matrix for A}
\NormalTok{A }\OtherTok{\textless{}{-}} \FunctionTok{matrix}\NormalTok{(}\FunctionTok{c}\NormalTok{(}\DecValTok{1}\SpecialCharTok{:}\DecValTok{9}\NormalTok{), }\AttributeTok{nrow =} \DecValTok{3}\NormalTok{, }\AttributeTok{ncol =} \DecValTok{3}\NormalTok{)}
\NormalTok{A}
\end{Highlighting}
\end{Shaded}

\begin{verbatim}
##      [,1] [,2] [,3]
## [1,]    1    4    7
## [2,]    2    5    8
## [3,]    3    6    9
\end{verbatim}

\begin{Shaded}
\begin{Highlighting}[]
\CommentTok{\# Create matrix for AT}
\NormalTok{AT }\OtherTok{\textless{}{-}} \FunctionTok{matrix}\NormalTok{(}\FunctionTok{c}\NormalTok{(}\DecValTok{1}\SpecialCharTok{:}\DecValTok{9}\NormalTok{), }\AttributeTok{nrow =} \DecValTok{3}\NormalTok{, }\AttributeTok{ncol =} \DecValTok{3}\NormalTok{)}
\NormalTok{AT}
\end{Highlighting}
\end{Shaded}

\begin{verbatim}
##      [,1] [,2] [,3]
## [1,]    1    4    7
## [2,]    2    5    8
## [3,]    3    6    9
\end{verbatim}

\(~\)

\begin{Shaded}
\begin{Highlighting}[]
\CommentTok{\# Multiply matrix A with matrix AT}
\NormalTok{A }\SpecialCharTok{\%*\%}\NormalTok{ AT}
\end{Highlighting}
\end{Shaded}

\begin{verbatim}
##      [,1] [,2] [,3]
## [1,]   30   66  102
## [2,]   36   81  126
## [3,]   42   96  150
\end{verbatim}

\begin{Shaded}
\begin{Highlighting}[]
\CommentTok{\# Multiply matrix AT with matrix A}
\NormalTok{AT }\SpecialCharTok{\%*\%}\NormalTok{ A}
\end{Highlighting}
\end{Shaded}

\begin{verbatim}
##      [,1] [,2] [,3]
## [1,]   30   66  102
## [2,]   36   81  126
## [3,]   42   96  150
\end{verbatim}

\(~\)

\begin{Shaded}
\begin{Highlighting}[]
\NormalTok{A }\SpecialCharTok{\%*\%}\NormalTok{ AT }\SpecialCharTok{==}\NormalTok{ AT }\SpecialCharTok{\%*\%}\NormalTok{ A}
\end{Highlighting}
\end{Shaded}

\begin{verbatim}
##      [,1] [,2] [,3]
## [1,] TRUE TRUE TRUE
## [2,] TRUE TRUE TRUE
## [3,] TRUE TRUE TRUE
\end{verbatim}

\(~\)

\hypertarget{conclusion}{%
\paragraph{Conclusion:}\label{conclusion}}

After transposing \texttt{ATA} matrix the results were a 3x3
\texttt{AAT} matrix, originally a 4x3, therefore in general,
\(A^TA \ne AA^T\). For part two, when the matrix is symmetrical the
transposing of the matrices equals to the matrix itself, therefore
\emph{\(A^TA = AA^T\)}.

\(~\)

\hypertarget{problem-set-2}{%
\subsubsection{Problem Set 2}\label{problem-set-2}}

\hypertarget{matrix-factorization-is-a-very-important-problem.-there-are-supercomputers-built-just-to-do-matrix-factorizations.-every-second-you-are-on-an-airplane-matrices-are-being-factorized.-radars-that-track-flights-use-a-technique-called-kalman-filtering.-at-the-heart-of-kalman-filtering-is-a-matrix-factorization-operation.-kalman-filters-are-solving-linear-systems-of-equations-when-they-track-your-flight-using-radars.-write-an-r-function-to-factorize-a-square-matrix-a-into-lu-or-ldu-whichever-you-prefer.-you-dont-have-to-worry-about-permuting-rows-of-a-and-you-can-assume-that-a-is-less-than-5x5-if-you-need-to-hard-code-any-variables-in-your-code.}{%
\paragraph{Matrix factorization is a very important problem. There are
supercomputers built just to do matrix factorizations. Every second you
are on an airplane, matrices are being factorized. Radars that track
flights use a technique called Kalman filtering. At the heart of Kalman
Filtering is a Matrix Factorization operation. Kalman Filters are
solving linear systems of equations when they track your flight using
radars. Write an R function to factorize a square matrix A into LU or
LDU, whichever you prefer. You don't have to worry about permuting rows
of A and you can assume that A is less than 5x5, if you need to
hard-code any variables in your
code.}\label{matrix-factorization-is-a-very-important-problem.-there-are-supercomputers-built-just-to-do-matrix-factorizations.-every-second-you-are-on-an-airplane-matrices-are-being-factorized.-radars-that-track-flights-use-a-technique-called-kalman-filtering.-at-the-heart-of-kalman-filtering-is-a-matrix-factorization-operation.-kalman-filters-are-solving-linear-systems-of-equations-when-they-track-your-flight-using-radars.-write-an-r-function-to-factorize-a-square-matrix-a-into-lu-or-ldu-whichever-you-prefer.-you-dont-have-to-worry-about-permuting-rows-of-a-and-you-can-assume-that-a-is-less-than-5x5-if-you-need-to-hard-code-any-variables-in-your-code.}}

\(~\)

\begin{Shaded}
\begin{Highlighting}[]
\CommentTok{\# Create the function}

\NormalTok{LU }\OtherTok{\textless{}{-}} \ControlFlowTok{function}\NormalTok{(A) \{}
 
\CommentTok{\# Upper triangular}
\NormalTok{  U }\OtherTok{\textless{}{-}}\NormalTok{ A}
  
\CommentTok{\# Setting matrix dimension}
\NormalTok{  n }\OtherTok{\textless{}{-}} \FunctionTok{dim}\NormalTok{(A)[}\DecValTok{1}\NormalTok{]}
  
\CommentTok{\# Lower triangular}
\NormalTok{  L }\OtherTok{\textless{}{-}} \FunctionTok{diag}\NormalTok{(n)}
  
  
  \ControlFlowTok{if}\NormalTok{ (n }\SpecialCharTok{==} \DecValTok{1}\NormalTok{) \{}
    \FunctionTok{return}\NormalTok{(}\FunctionTok{list}\NormalTok{(L, U))}
\NormalTok{  \}}
  
  \ControlFlowTok{for}\NormalTok{(a }\ControlFlowTok{in} \DecValTok{2}\SpecialCharTok{:}\NormalTok{n) \{}
    \ControlFlowTok{for}\NormalTok{(t }\ControlFlowTok{in} \DecValTok{1}\SpecialCharTok{:}\NormalTok{(a }\SpecialCharTok{{-}} \DecValTok{1}\NormalTok{)) \{}
\NormalTok{      multiplier }\OtherTok{\textless{}{-}} \SpecialCharTok{{-}}\NormalTok{U[a, t] }\SpecialCharTok{/}\NormalTok{ U[t, t]}
\NormalTok{      U[a, ] }\OtherTok{\textless{}{-}}\NormalTok{ multiplier }\SpecialCharTok{*}\NormalTok{ U[t, ] }\SpecialCharTok{+}\NormalTok{ U[a, ]}
\NormalTok{      L[a, t] }\OtherTok{\textless{}{-}} \SpecialCharTok{{-}}\NormalTok{multiplier}
\NormalTok{    \}}
\NormalTok{  \}}
  \FunctionTok{return}\NormalTok{(}\FunctionTok{list}\NormalTok{(L,U))}
\NormalTok{\}}
\end{Highlighting}
\end{Shaded}

\(~\)

\begin{Shaded}
\begin{Highlighting}[]
\CommentTok{\# Application to the function}

\CommentTok{\# Testing with a 3x3 matrix}
\NormalTok{A }\OtherTok{\textless{}{-}} \FunctionTok{matrix}\NormalTok{(}\DecValTok{1}\SpecialCharTok{:}\DecValTok{9}\NormalTok{, }\AttributeTok{nrow =} \DecValTok{3}\NormalTok{, }\AttributeTok{byrow =}\NormalTok{ T)}
\NormalTok{A}
\end{Highlighting}
\end{Shaded}

\begin{verbatim}
##      [,1] [,2] [,3]
## [1,]    1    2    3
## [2,]    4    5    6
## [3,]    7    8    9
\end{verbatim}

\begin{Shaded}
\begin{Highlighting}[]
\NormalTok{LU\_2 }\OtherTok{\textless{}{-}} \FunctionTok{LU}\NormalTok{(A)}

\CommentTok{\# Multiply the upper and lower matrix}
\NormalTok{lower\_multiply }\OtherTok{\textless{}{-}}\NormalTok{ LU\_2[[}\DecValTok{1}\NormalTok{]]}
\NormalTok{lower\_multiply}
\end{Highlighting}
\end{Shaded}

\begin{verbatim}
##      [,1] [,2] [,3]
## [1,]    1    0    0
## [2,]    4    1    0
## [3,]    7    2    1
\end{verbatim}

\begin{Shaded}
\begin{Highlighting}[]
\NormalTok{upper\_multiply }\OtherTok{\textless{}{-}}\NormalTok{ LU\_2[[}\DecValTok{2}\NormalTok{]]}
\NormalTok{upper\_multiply}
\end{Highlighting}
\end{Shaded}

\begin{verbatim}
##      [,1] [,2] [,3]
## [1,]    1    2    3
## [2,]    0   -3   -6
## [3,]    0    0    0
\end{verbatim}

\begin{Shaded}
\begin{Highlighting}[]
\NormalTok{A }\SpecialCharTok{==}\NormalTok{ lower\_multiply }\SpecialCharTok{\%*\%}\NormalTok{ upper\_multiply}
\end{Highlighting}
\end{Shaded}

\begin{verbatim}
##      [,1] [,2] [,3]
## [1,] TRUE TRUE TRUE
## [2,] TRUE TRUE TRUE
## [3,] TRUE TRUE TRUE
\end{verbatim}

\(~\)

\hypertarget{references}{%
\subsubsection{References:}\label{references}}

\begin{itemize}
\tightlist
\item
  \url{https://www.geeksforgeeks.org/doolittle-algorithm-lu-decomposition/}
\end{itemize}

\end{document}
